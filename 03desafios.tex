% Options for packages loaded elsewhere
\PassOptionsToPackage{unicode}{hyperref}
\PassOptionsToPackage{hyphens}{url}
%
\documentclass[
]{article}
\usepackage{amsmath,amssymb}
\usepackage{lmodern}
\usepackage{iftex}
\ifPDFTeX
  \usepackage[T1]{fontenc}
  \usepackage[utf8]{inputenc}
  \usepackage{textcomp} % provide euro and other symbols
\else % if luatex or xetex
  \usepackage{unicode-math}
  \defaultfontfeatures{Scale=MatchLowercase}
  \defaultfontfeatures[\rmfamily]{Ligatures=TeX,Scale=1}
\fi
% Use upquote if available, for straight quotes in verbatim environments
\IfFileExists{upquote.sty}{\usepackage{upquote}}{}
\IfFileExists{microtype.sty}{% use microtype if available
  \usepackage[]{microtype}
  \UseMicrotypeSet[protrusion]{basicmath} % disable protrusion for tt fonts
}{}
\makeatletter
\@ifundefined{KOMAClassName}{% if non-KOMA class
  \IfFileExists{parskip.sty}{%
    \usepackage{parskip}
  }{% else
    \setlength{\parindent}{0pt}
    \setlength{\parskip}{6pt plus 2pt minus 1pt}}
}{% if KOMA class
  \KOMAoptions{parskip=half}}
\makeatother
\usepackage{xcolor}
\usepackage[margin=1in]{geometry}
\usepackage{color}
\usepackage{fancyvrb}
\newcommand{\VerbBar}{|}
\newcommand{\VERB}{\Verb[commandchars=\\\{\}]}
\DefineVerbatimEnvironment{Highlighting}{Verbatim}{commandchars=\\\{\}}
% Add ',fontsize=\small' for more characters per line
\usepackage{framed}
\definecolor{shadecolor}{RGB}{248,248,248}
\newenvironment{Shaded}{\begin{snugshade}}{\end{snugshade}}
\newcommand{\AlertTok}[1]{\textcolor[rgb]{0.94,0.16,0.16}{#1}}
\newcommand{\AnnotationTok}[1]{\textcolor[rgb]{0.56,0.35,0.01}{\textbf{\textit{#1}}}}
\newcommand{\AttributeTok}[1]{\textcolor[rgb]{0.77,0.63,0.00}{#1}}
\newcommand{\BaseNTok}[1]{\textcolor[rgb]{0.00,0.00,0.81}{#1}}
\newcommand{\BuiltInTok}[1]{#1}
\newcommand{\CharTok}[1]{\textcolor[rgb]{0.31,0.60,0.02}{#1}}
\newcommand{\CommentTok}[1]{\textcolor[rgb]{0.56,0.35,0.01}{\textit{#1}}}
\newcommand{\CommentVarTok}[1]{\textcolor[rgb]{0.56,0.35,0.01}{\textbf{\textit{#1}}}}
\newcommand{\ConstantTok}[1]{\textcolor[rgb]{0.00,0.00,0.00}{#1}}
\newcommand{\ControlFlowTok}[1]{\textcolor[rgb]{0.13,0.29,0.53}{\textbf{#1}}}
\newcommand{\DataTypeTok}[1]{\textcolor[rgb]{0.13,0.29,0.53}{#1}}
\newcommand{\DecValTok}[1]{\textcolor[rgb]{0.00,0.00,0.81}{#1}}
\newcommand{\DocumentationTok}[1]{\textcolor[rgb]{0.56,0.35,0.01}{\textbf{\textit{#1}}}}
\newcommand{\ErrorTok}[1]{\textcolor[rgb]{0.64,0.00,0.00}{\textbf{#1}}}
\newcommand{\ExtensionTok}[1]{#1}
\newcommand{\FloatTok}[1]{\textcolor[rgb]{0.00,0.00,0.81}{#1}}
\newcommand{\FunctionTok}[1]{\textcolor[rgb]{0.00,0.00,0.00}{#1}}
\newcommand{\ImportTok}[1]{#1}
\newcommand{\InformationTok}[1]{\textcolor[rgb]{0.56,0.35,0.01}{\textbf{\textit{#1}}}}
\newcommand{\KeywordTok}[1]{\textcolor[rgb]{0.13,0.29,0.53}{\textbf{#1}}}
\newcommand{\NormalTok}[1]{#1}
\newcommand{\OperatorTok}[1]{\textcolor[rgb]{0.81,0.36,0.00}{\textbf{#1}}}
\newcommand{\OtherTok}[1]{\textcolor[rgb]{0.56,0.35,0.01}{#1}}
\newcommand{\PreprocessorTok}[1]{\textcolor[rgb]{0.56,0.35,0.01}{\textit{#1}}}
\newcommand{\RegionMarkerTok}[1]{#1}
\newcommand{\SpecialCharTok}[1]{\textcolor[rgb]{0.00,0.00,0.00}{#1}}
\newcommand{\SpecialStringTok}[1]{\textcolor[rgb]{0.31,0.60,0.02}{#1}}
\newcommand{\StringTok}[1]{\textcolor[rgb]{0.31,0.60,0.02}{#1}}
\newcommand{\VariableTok}[1]{\textcolor[rgb]{0.00,0.00,0.00}{#1}}
\newcommand{\VerbatimStringTok}[1]{\textcolor[rgb]{0.31,0.60,0.02}{#1}}
\newcommand{\WarningTok}[1]{\textcolor[rgb]{0.56,0.35,0.01}{\textbf{\textit{#1}}}}
\usepackage{graphicx}
\makeatletter
\def\maxwidth{\ifdim\Gin@nat@width>\linewidth\linewidth\else\Gin@nat@width\fi}
\def\maxheight{\ifdim\Gin@nat@height>\textheight\textheight\else\Gin@nat@height\fi}
\makeatother
% Scale images if necessary, so that they will not overflow the page
% margins by default, and it is still possible to overwrite the defaults
% using explicit options in \includegraphics[width, height, ...]{}
\setkeys{Gin}{width=\maxwidth,height=\maxheight,keepaspectratio}
% Set default figure placement to htbp
\makeatletter
\def\fps@figure{htbp}
\makeatother
\setlength{\emergencystretch}{3em} % prevent overfull lines
\providecommand{\tightlist}{%
  \setlength{\itemsep}{0pt}\setlength{\parskip}{0pt}}
\setcounter{secnumdepth}{-\maxdimen} % remove section numbering
\newlength{\cslhangindent}
\setlength{\cslhangindent}{1.5em}
\newlength{\csllabelwidth}
\setlength{\csllabelwidth}{3em}
\newlength{\cslentryspacingunit} % times entry-spacing
\setlength{\cslentryspacingunit}{\parskip}
\newenvironment{CSLReferences}[2] % #1 hanging-ident, #2 entry spacing
 {% don't indent paragraphs
  \setlength{\parindent}{0pt}
  % turn on hanging indent if param 1 is 1
  \ifodd #1
  \let\oldpar\par
  \def\par{\hangindent=\cslhangindent\oldpar}
  \fi
  % set entry spacing
  \setlength{\parskip}{#2\cslentryspacingunit}
 }%
 {}
\usepackage{calc}
\newcommand{\CSLBlock}[1]{#1\hfill\break}
\newcommand{\CSLLeftMargin}[1]{\parbox[t]{\csllabelwidth}{#1}}
\newcommand{\CSLRightInline}[1]{\parbox[t]{\linewidth - \csllabelwidth}{#1}\break}
\newcommand{\CSLIndent}[1]{\hspace{\cslhangindent}#1}
\ifLuaTeX
  \usepackage{selnolig}  % disable illegal ligatures
\fi
\IfFileExists{bookmark.sty}{\usepackage{bookmark}}{\usepackage{hyperref}}
\IfFileExists{xurl.sty}{\usepackage{xurl}}{} % add URL line breaks if available
\urlstyle{same} % disable monospaced font for URLs
\hypersetup{
  pdftitle={03desafios},
  pdfauthor={Valentina Paz Campos Olguín},
  hidelinks,
  pdfcreator={LaTeX via pandoc}}

\title{03desafios}
\author{Valentina Paz Campos Olguín}
\date{2023-05-26}

\begin{document}
\maketitle

\hypertarget{ejercicio-3}{%
\section{Ejercicio 3}\label{ejercicio-3}}

\hypertarget{explique-para-quuxe9-se-puede-utilizar-o-se-ha-utilizado-cada-distribuciuxf3n.-cite-al-menos-un-artuxedculo-cientuxedfico.}{%
\subsection{1. Explique para qué se puede utilizar o se ha utilizado
cada distribución. Cite al menos un artículo
científico.}\label{explique-para-quuxe9-se-puede-utilizar-o-se-ha-utilizado-cada-distribuciuxf3n.-cite-al-menos-un-artuxedculo-cientuxedfico.}}

\hypertarget{chi-squared}{%
\subsubsection{Chi-squared}\label{chi-squared}}

Es una distribución de probabilidad que sirve para realizar pruebas de
independencia y construir intervalos de confianza para la varianza. Se
ha utilizado en la estadística inferencial, investigación social e
investigación médica.

Por ejemplo, en el año 2007 se hizo una investigación pedriátrica para
establecer si existía o no una independencia entre las variables
cualitativas de los pacientes tales como: el sexo del recién nacido o su
grado de desnutrición (Cerda \& VILLARROEL DEL, 2007).

\hypertarget{inverse-gamma}{%
\subsubsection{Inverse gamma}\label{inverse-gamma}}

Es una distribución de probabilidad que sirve para modelar variables
aleatorias positivas con asimetría hacia la derecha. Se ha utilizado
para realizar aplicaciones en el modelado de tiempos de vida, análisis
de fiabilidad y de supervivencia. Describe y analiza variables
aleatorias continuas no negativas.

Por ejemplo, según (2015) se utiliza también para resolver problemas de
la teoría de difracción y de corrosión en computadoras nuevas.

\hypertarget{student}{%
\subsubsection{Student}\label{student}}

Es una distribución de probabilidad que sirve para hacer inferencias
sobre el promedio de una población cuando se extrae una pequeña muestra.
Gracias a esto, se pueden realizar hipótesis, descubrir incertidumbres
que proporcionen intervalos de confianza e inferir información a partir
de los parámetros de la población.

Value at risk (VaR) podría ser la medida más conocida para medir el
riesgo de mercado. En este estudio se demuestra que la distribución
Student es una de las más adecuadas para tratar este tipo de valores
(Gao \& Zhou, 2016).

\hypertarget{graficar-las-funciones-de-probabilidad-y-de-probabilidad-acumulada-de-cada-distribuciuxf3n.}{%
\subsection{2. Graficar las funciones de probabilidad y de probabilidad
acumulada de cada
distribución.}\label{graficar-las-funciones-de-probabilidad-y-de-probabilidad-acumulada-de-cada-distribuciuxf3n.}}

\hypertarget{chi-squared-1}{%
\subsubsection{Chi-squared}\label{chi-squared-1}}

\begin{Shaded}
\begin{Highlighting}[]
\FunctionTok{library}\NormalTok{(ggplot2)}
\NormalTok{x }\OtherTok{=} \FunctionTok{seq}\NormalTok{(}\DecValTok{0}\NormalTok{, }\DecValTok{10}\NormalTok{, }\AttributeTok{length =} \DecValTok{1000}\NormalTok{)}


\NormalTok{colors1 }\OtherTok{\textless{}{-}} \FunctionTok{c}\NormalTok{(}\StringTok{"k1 = 1"} \OtherTok{=} \StringTok{"blue4"}\NormalTok{, }\StringTok{"k2 = 2"} \OtherTok{=} \StringTok{"red4"}\NormalTok{, }\StringTok{"k3 = 3"} \OtherTok{=} \StringTok{"green4"}\NormalTok{, }\StringTok{"k4 = 5"} \OtherTok{=} \StringTok{"coral3"}\NormalTok{, }\StringTok{"k5 = 10"} \OtherTok{=} \StringTok{"yellow3"}\NormalTok{)}

\NormalTok{distribucion }\OtherTok{=} \FunctionTok{dchisq}\NormalTok{(x, }\AttributeTok{df =} \DecValTok{1}\NormalTok{, }\AttributeTok{ncp =} \DecValTok{0}\NormalTok{, }\AttributeTok{log =} \ConstantTok{FALSE}\NormalTok{)}
\NormalTok{distribucion2 }\OtherTok{=} \FunctionTok{dchisq}\NormalTok{(x, }\AttributeTok{df =} \DecValTok{2}\NormalTok{, }\AttributeTok{ncp =} \DecValTok{0}\NormalTok{, }\AttributeTok{log =} \ConstantTok{FALSE}\NormalTok{)}
\NormalTok{distribucion3 }\OtherTok{=} \FunctionTok{dchisq}\NormalTok{(x, }\AttributeTok{df =} \DecValTok{3}\NormalTok{, }\AttributeTok{ncp =} \DecValTok{0}\NormalTok{, }\AttributeTok{log =} \ConstantTok{FALSE}\NormalTok{)}
\NormalTok{distribucion5 }\OtherTok{=} \FunctionTok{dchisq}\NormalTok{(x, }\AttributeTok{df =} \DecValTok{5}\NormalTok{, }\AttributeTok{ncp =} \DecValTok{0}\NormalTok{, }\AttributeTok{log =} \ConstantTok{FALSE}\NormalTok{)}
\NormalTok{distribucion10 }\OtherTok{=} \FunctionTok{dchisq}\NormalTok{(x, }\AttributeTok{df =} \DecValTok{10}\NormalTok{, }\AttributeTok{ncp =} \DecValTok{0}\NormalTok{, }\AttributeTok{log =} \ConstantTok{FALSE}\NormalTok{)}

\NormalTok{datos}\OtherTok{=}\FunctionTok{data.frame}\NormalTok{(x,distribucion)}

\NormalTok{grafico }\OtherTok{=} \FunctionTok{ggplot}\NormalTok{(}\AttributeTok{data=}\NormalTok{datos,}\FunctionTok{aes}\NormalTok{(}\AttributeTok{x=}\NormalTok{x,}\AttributeTok{y=}\NormalTok{distribucion))}
\NormalTok{grafico }\OtherTok{=}\NormalTok{ grafico }\SpecialCharTok{+} \FunctionTok{geom\_line}\NormalTok{(}\FunctionTok{aes}\NormalTok{(}\AttributeTok{y=}\NormalTok{distribucion, }\AttributeTok{color =} \StringTok{"k1 = 1"}\NormalTok{), }\AttributeTok{stat=}\StringTok{"identity"}\NormalTok{, }\AttributeTok{linewidth =} \DecValTok{1}\NormalTok{)}
\NormalTok{grafico }\OtherTok{=}\NormalTok{ grafico }\SpecialCharTok{+} \FunctionTok{geom\_line}\NormalTok{(}\FunctionTok{aes}\NormalTok{(}\AttributeTok{y=}\NormalTok{distribucion2, }\AttributeTok{color =} \StringTok{"k2 = 2"}\NormalTok{), }\AttributeTok{stat=}\StringTok{"identity"}\NormalTok{, }\AttributeTok{linewidth =} \DecValTok{1}\NormalTok{)}
\NormalTok{grafico }\OtherTok{=}\NormalTok{ grafico }\SpecialCharTok{+} \FunctionTok{geom\_line}\NormalTok{(}\FunctionTok{aes}\NormalTok{(}\AttributeTok{y=}\NormalTok{distribucion3, }\AttributeTok{color =} \StringTok{"k3 = 3"}\NormalTok{), }\AttributeTok{stat=}\StringTok{"identity"}\NormalTok{, }\AttributeTok{linewidth =} \DecValTok{1}\NormalTok{)}
\NormalTok{grafico }\OtherTok{=}\NormalTok{ grafico }\SpecialCharTok{+} \FunctionTok{geom\_line}\NormalTok{(}\FunctionTok{aes}\NormalTok{(}\AttributeTok{y=}\NormalTok{distribucion5, }\AttributeTok{color =} \StringTok{"k4 = 5"}\NormalTok{), }\AttributeTok{stat=}\StringTok{"identity"}\NormalTok{, }\AttributeTok{linewidth =} \DecValTok{1}\NormalTok{)}
\NormalTok{grafico }\OtherTok{=}\NormalTok{ grafico }\SpecialCharTok{+} \FunctionTok{geom\_line}\NormalTok{(}\FunctionTok{aes}\NormalTok{(}\AttributeTok{y=}\NormalTok{distribucion10, }\AttributeTok{color =} \StringTok{"k5 = 10"}\NormalTok{), }\AttributeTok{stat=}\StringTok{"identity"}\NormalTok{, }\AttributeTok{linewidth =} \DecValTok{1}\NormalTok{)}
\NormalTok{grafico }\OtherTok{=}\NormalTok{ grafico }\SpecialCharTok{+} \FunctionTok{theme\_bw}\NormalTok{() }\SpecialCharTok{+} \FunctionTok{ggtitle}\NormalTok{(}\StringTok{"Función de probabilidad {-} Chi{-}square"}\NormalTok{)}
\NormalTok{grafico }\OtherTok{=}\NormalTok{ grafico }\SpecialCharTok{+} \FunctionTok{xlab}\NormalTok{(}\StringTok{"X"}\NormalTok{) }\SpecialCharTok{+} \FunctionTok{ylab}\NormalTok{(}\StringTok{"Probabilidad"}\NormalTok{)}
\NormalTok{grafico }\OtherTok{=}\NormalTok{ grafico }\SpecialCharTok{+} \FunctionTok{coord\_cartesian}\NormalTok{(}\AttributeTok{ylim=}\FunctionTok{c}\NormalTok{(}\DecValTok{0}\NormalTok{,}\DecValTok{1}\NormalTok{))}
\NormalTok{grafico }\OtherTok{=}\NormalTok{ grafico }\SpecialCharTok{+} \FunctionTok{labs}\NormalTok{(}\AttributeTok{color =} \StringTok{"Valor de k"}\NormalTok{)}
\FunctionTok{plot}\NormalTok{(grafico)}
\end{Highlighting}
\end{Shaded}

\includegraphics{03desafios_files/figure-latex/unnamed-chunk-1-1.pdf}

\begin{Shaded}
\begin{Highlighting}[]
\NormalTok{x }\OtherTok{=} \FunctionTok{seq}\NormalTok{(}\DecValTok{0}\NormalTok{, }\DecValTok{10}\NormalTok{, }\AttributeTok{length =} \DecValTok{1000}\NormalTok{)}

\NormalTok{distribucion }\OtherTok{=} \FunctionTok{pchisq}\NormalTok{(x, }\AttributeTok{df =} \DecValTok{2}\NormalTok{, }\AttributeTok{lower.tail =} \ConstantTok{TRUE}\NormalTok{, }\AttributeTok{log.p =} \ConstantTok{FALSE}\NormalTok{)}

\NormalTok{datos}\OtherTok{=}\FunctionTok{data.frame}\NormalTok{(x,distribucion)}

\NormalTok{grafico }\OtherTok{=} \FunctionTok{ggplot}\NormalTok{(}\AttributeTok{data=}\NormalTok{datos,}\FunctionTok{aes}\NormalTok{(}\AttributeTok{x=}\NormalTok{x,}\AttributeTok{y=}\NormalTok{distribucion))}
\NormalTok{grafico }\OtherTok{=}\NormalTok{ grafico }\SpecialCharTok{+} \FunctionTok{geom\_line}\NormalTok{(}\AttributeTok{stat=}\StringTok{"identity"}\NormalTok{, }\AttributeTok{color=}\StringTok{"blue4"}\NormalTok{, }\AttributeTok{linewidth =} \DecValTok{1}\NormalTok{)}
\NormalTok{grafico }\OtherTok{=}\NormalTok{ grafico }\SpecialCharTok{+} \FunctionTok{theme\_bw}\NormalTok{() }\SpecialCharTok{+} \FunctionTok{ggtitle}\NormalTok{(}\StringTok{"Probabilidad acumulada {-} Chi{-}square"}\NormalTok{)}
\NormalTok{grafico }\OtherTok{=}\NormalTok{ grafico }\SpecialCharTok{+} \FunctionTok{xlab}\NormalTok{(}\StringTok{"X"}\NormalTok{) }\SpecialCharTok{+} \FunctionTok{ylab}\NormalTok{(}\StringTok{"Probabilidad"}\NormalTok{)}
\FunctionTok{plot}\NormalTok{(grafico)}
\end{Highlighting}
\end{Shaded}

\includegraphics{03desafios_files/figure-latex/unnamed-chunk-2-1.pdf}

\hypertarget{inverse-gamma-1}{%
\subsubsection{Inverse Gamma}\label{inverse-gamma-1}}

\begin{Shaded}
\begin{Highlighting}[]
\FunctionTok{library}\NormalTok{(extraDistr)}

\NormalTok{colors2 }\OtherTok{\textless{}{-}} \FunctionTok{c}\NormalTok{(}\StringTok{"alpha = 1, beta = 1"} \OtherTok{=} \StringTok{"blue4"}\NormalTok{, }\StringTok{"alpha = 2, beta = 1"} \OtherTok{=} \StringTok{"red4"}\NormalTok{, }\StringTok{"alpha = 3, beta = 1"} \OtherTok{=} \StringTok{"green4"}\NormalTok{, }\StringTok{"alpha = 3, beta = 0.5"} \OtherTok{=} \StringTok{"yellow3"}\NormalTok{)}

\NormalTok{x }\OtherTok{=} \FunctionTok{seq}\NormalTok{(}\DecValTok{0}\NormalTok{, }\DecValTok{3}\NormalTok{, }\AttributeTok{length =} \DecValTok{1000}\NormalTok{)}
\NormalTok{distribucion }\OtherTok{=} \FunctionTok{dinvgamma}\NormalTok{(x, }\AttributeTok{alpha =} \DecValTok{1}\NormalTok{, }\AttributeTok{beta =} \DecValTok{1}\NormalTok{, }\AttributeTok{log =} \ConstantTok{FALSE}\NormalTok{)}
\NormalTok{distribucion2 }\OtherTok{=} \FunctionTok{dinvgamma}\NormalTok{(x, }\AttributeTok{alpha =} \DecValTok{2}\NormalTok{, }\AttributeTok{beta =} \DecValTok{1}\NormalTok{, }\AttributeTok{log =} \ConstantTok{FALSE}\NormalTok{)}
\NormalTok{distribucion3 }\OtherTok{=} \FunctionTok{dinvgamma}\NormalTok{(x, }\AttributeTok{alpha =} \DecValTok{3}\NormalTok{, }\AttributeTok{beta =} \DecValTok{1}\NormalTok{, }\AttributeTok{log =} \ConstantTok{FALSE}\NormalTok{)}
\NormalTok{distribucion4 }\OtherTok{=} \FunctionTok{dinvgamma}\NormalTok{(x, }\AttributeTok{alpha =} \DecValTok{3}\NormalTok{, }\AttributeTok{beta =} \FloatTok{0.5}\NormalTok{, }\AttributeTok{log =} \ConstantTok{FALSE}\NormalTok{)}

\NormalTok{datos}\OtherTok{=}\FunctionTok{data.frame}\NormalTok{(x,distribucion)}

\NormalTok{grafico }\OtherTok{=} \FunctionTok{ggplot}\NormalTok{(}\AttributeTok{data=}\NormalTok{datos,}\FunctionTok{aes}\NormalTok{(}\AttributeTok{x=}\NormalTok{x,}\AttributeTok{y=}\NormalTok{distribucion))}
\NormalTok{grafico }\OtherTok{=}\NormalTok{ grafico }\SpecialCharTok{+} \FunctionTok{geom\_line}\NormalTok{(}\FunctionTok{aes}\NormalTok{(}\AttributeTok{y=}\NormalTok{distribucion, }\AttributeTok{color =} \StringTok{"alpha = 1, beta = 1"}\NormalTok{), }\AttributeTok{stat=}\StringTok{"identity"}\NormalTok{, }\AttributeTok{linewidth =} \DecValTok{1}\NormalTok{)}
\NormalTok{grafico }\OtherTok{=}\NormalTok{ grafico }\SpecialCharTok{+} \FunctionTok{geom\_line}\NormalTok{(}\FunctionTok{aes}\NormalTok{(}\AttributeTok{y=}\NormalTok{distribucion2, }\AttributeTok{color =} \StringTok{"alpha = 2, beta = 1"}\NormalTok{), }\AttributeTok{stat=}\StringTok{"identity"}\NormalTok{, }\AttributeTok{linewidth =} \DecValTok{1}\NormalTok{)}
\NormalTok{grafico }\OtherTok{=}\NormalTok{ grafico }\SpecialCharTok{+} \FunctionTok{geom\_line}\NormalTok{(}\FunctionTok{aes}\NormalTok{(}\AttributeTok{y=}\NormalTok{distribucion3, }\AttributeTok{color =} \StringTok{"alpha = 3, beta = 1"}\NormalTok{), }\AttributeTok{stat=}\StringTok{"identity"}\NormalTok{, }\AttributeTok{linewidth =} \DecValTok{1}\NormalTok{)}
\NormalTok{grafico }\OtherTok{=}\NormalTok{ grafico }\SpecialCharTok{+} \FunctionTok{geom\_line}\NormalTok{(}\FunctionTok{aes}\NormalTok{(}\AttributeTok{y=}\NormalTok{distribucion4, }\AttributeTok{color =} \StringTok{"alpha = 3, beta = 0.5"}\NormalTok{), }\AttributeTok{stat=}\StringTok{"identity"}\NormalTok{, }\AttributeTok{linewidth =} \DecValTok{1}\NormalTok{)}
\NormalTok{grafico }\OtherTok{=}\NormalTok{ grafico }\SpecialCharTok{+} \FunctionTok{theme\_bw}\NormalTok{() }\SpecialCharTok{+} \FunctionTok{ggtitle}\NormalTok{(}\StringTok{"Distribución de probabilidades {-} Inverse Gamma"}\NormalTok{)}
\NormalTok{grafico }\OtherTok{=}\NormalTok{ grafico }\SpecialCharTok{+} \FunctionTok{xlab}\NormalTok{(}\StringTok{"X"}\NormalTok{) }\SpecialCharTok{+} \FunctionTok{ylab}\NormalTok{(}\StringTok{"Probabilidad"}\NormalTok{)}
\NormalTok{grafico }\OtherTok{=}\NormalTok{ grafico }\SpecialCharTok{+} \FunctionTok{labs}\NormalTok{(}\AttributeTok{color =} \StringTok{"Valores de alpha y beta"}\NormalTok{)}

\FunctionTok{plot}\NormalTok{(grafico)}
\end{Highlighting}
\end{Shaded}

\includegraphics{03desafios_files/figure-latex/unnamed-chunk-3-1.pdf}

\begin{Shaded}
\begin{Highlighting}[]
\NormalTok{x }\OtherTok{=} \FunctionTok{seq}\NormalTok{(}\DecValTok{0}\NormalTok{, }\DecValTok{3}\NormalTok{, }\AttributeTok{length =} \DecValTok{1000}\NormalTok{)}
\NormalTok{alpha }\OtherTok{=} \DecValTok{3}
\NormalTok{distribucion }\OtherTok{=} \FunctionTok{pinvgamma}\NormalTok{(x, alpha, }\AttributeTok{beta =} \FloatTok{0.5}\NormalTok{, }\AttributeTok{log =} \ConstantTok{FALSE}\NormalTok{)}

\NormalTok{datos}\OtherTok{=}\FunctionTok{data.frame}\NormalTok{(x,distribucion)}

\NormalTok{grafico }\OtherTok{=} \FunctionTok{ggplot}\NormalTok{(}\AttributeTok{data=}\NormalTok{datos,}\FunctionTok{aes}\NormalTok{(}\AttributeTok{x=}\NormalTok{x,}\AttributeTok{y=}\NormalTok{distribucion))}
\NormalTok{grafico }\OtherTok{=}\NormalTok{ grafico }\SpecialCharTok{+} \FunctionTok{geom\_line}\NormalTok{(}\AttributeTok{stat=}\StringTok{"identity"}\NormalTok{, }\AttributeTok{color=}\StringTok{"blue4"}\NormalTok{, }\AttributeTok{linewidth =} \DecValTok{1}\NormalTok{)}
\NormalTok{grafico }\OtherTok{=}\NormalTok{ grafico }\SpecialCharTok{+} \FunctionTok{theme\_bw}\NormalTok{() }\SpecialCharTok{+} \FunctionTok{ggtitle}\NormalTok{(}\StringTok{"Probabilidad acumulada {-} Inverse Gamma"}\NormalTok{)}
\NormalTok{grafico }\OtherTok{=}\NormalTok{ grafico }\SpecialCharTok{+} \FunctionTok{xlab}\NormalTok{(}\StringTok{"X"}\NormalTok{) }\SpecialCharTok{+} \FunctionTok{ylab}\NormalTok{(}\StringTok{"Probabilidad"}\NormalTok{)}
\FunctionTok{plot}\NormalTok{(grafico)}
\end{Highlighting}
\end{Shaded}

\includegraphics{03desafios_files/figure-latex/unnamed-chunk-4-1.pdf}

\hypertarget{student-1}{%
\subsubsection{Student}\label{student-1}}

\begin{Shaded}
\begin{Highlighting}[]
\NormalTok{x }\OtherTok{=} \FunctionTok{seq}\NormalTok{(}\SpecialCharTok{{-}}\DecValTok{4}\NormalTok{, }\DecValTok{4}\NormalTok{, }\AttributeTok{length =} \DecValTok{100}\NormalTok{)}
\NormalTok{distribucion }\OtherTok{=} \FunctionTok{dt}\NormalTok{(x, }\AttributeTok{df =} \DecValTok{1}\NormalTok{, }\AttributeTok{log =} \ConstantTok{FALSE}\NormalTok{)}
\NormalTok{distribucion2 }\OtherTok{=} \FunctionTok{dt}\NormalTok{(x, }\AttributeTok{df =} \DecValTok{2}\NormalTok{, }\AttributeTok{log =} \ConstantTok{FALSE}\NormalTok{)}
\NormalTok{distribucion5 }\OtherTok{=} \FunctionTok{dt}\NormalTok{(x, }\AttributeTok{df =} \DecValTok{5}\NormalTok{, }\AttributeTok{log =} \ConstantTok{FALSE}\NormalTok{)}
\NormalTok{distribucion10 }\OtherTok{=} \FunctionTok{dt}\NormalTok{(x, }\AttributeTok{df =} \DecValTok{1000000}\NormalTok{, }\AttributeTok{log =} \ConstantTok{FALSE}\NormalTok{)}

\NormalTok{colors3 }\OtherTok{\textless{}{-}} \FunctionTok{c}\NormalTok{(}\StringTok{"df1 = 1"} \OtherTok{=} \StringTok{"blue4"}\NormalTok{, }\StringTok{"df2 = 2"} \OtherTok{=} \StringTok{"red4"}\NormalTok{, }\StringTok{"df3 = 5"} \OtherTok{=} \StringTok{"green4"}\NormalTok{, }\StringTok{"df4 = 1000000"} \OtherTok{=} \StringTok{"yellow3"}\NormalTok{)}

\NormalTok{datos}\OtherTok{=}\FunctionTok{data.frame}\NormalTok{(x,distribucion)}

\NormalTok{grafico }\OtherTok{=} \FunctionTok{ggplot}\NormalTok{(}\AttributeTok{data=}\NormalTok{datos,}\FunctionTok{aes}\NormalTok{(}\AttributeTok{x=}\NormalTok{x,}\AttributeTok{y=}\NormalTok{distribucion))}
\NormalTok{grafico }\OtherTok{=}\NormalTok{ grafico }\SpecialCharTok{+} \FunctionTok{geom\_line}\NormalTok{(}\FunctionTok{aes}\NormalTok{(}\AttributeTok{y=}\NormalTok{distribucion, }\AttributeTok{color =} \StringTok{"df1 = 1"}\NormalTok{), }\AttributeTok{stat=}\StringTok{"identity"}\NormalTok{, }\AttributeTok{linewidth =} \DecValTok{1}\NormalTok{)}
\NormalTok{grafico }\OtherTok{=}\NormalTok{ grafico }\SpecialCharTok{+} \FunctionTok{geom\_line}\NormalTok{(}\FunctionTok{aes}\NormalTok{(}\AttributeTok{y=}\NormalTok{distribucion2, }\AttributeTok{color =} \StringTok{"df2 = 2"}\NormalTok{), }\AttributeTok{stat=}\StringTok{"identity"}\NormalTok{, }\AttributeTok{linewidth =} \DecValTok{1}\NormalTok{)}
\NormalTok{grafico }\OtherTok{=}\NormalTok{ grafico }\SpecialCharTok{+} \FunctionTok{geom\_line}\NormalTok{(}\FunctionTok{aes}\NormalTok{(}\AttributeTok{y=}\NormalTok{distribucion5, }\AttributeTok{color =} \StringTok{"df3 = 5"}\NormalTok{), }\AttributeTok{stat=}\StringTok{"identity"}\NormalTok{, }\AttributeTok{linewidth =} \DecValTok{1}\NormalTok{)}
\NormalTok{grafico }\OtherTok{=}\NormalTok{ grafico }\SpecialCharTok{+} \FunctionTok{geom\_line}\NormalTok{(}\FunctionTok{aes}\NormalTok{(}\AttributeTok{y=}\NormalTok{distribucion10, }\AttributeTok{color =} \StringTok{"df4 = 1000000"}\NormalTok{), }\AttributeTok{stat=}\StringTok{"identity"}\NormalTok{, }\AttributeTok{linewidth =} \DecValTok{1}\NormalTok{)}
\NormalTok{grafico }\OtherTok{=}\NormalTok{ grafico }\SpecialCharTok{+} \FunctionTok{theme\_bw}\NormalTok{() }\SpecialCharTok{+} \FunctionTok{ggtitle}\NormalTok{(}\StringTok{"Distribución de probabilidades {-} Student"}\NormalTok{)}
\NormalTok{grafico }\OtherTok{=}\NormalTok{ grafico }\SpecialCharTok{+} \FunctionTok{xlab}\NormalTok{(}\StringTok{"X"}\NormalTok{) }\SpecialCharTok{+} \FunctionTok{ylab}\NormalTok{(}\StringTok{"Probabilidad"}\NormalTok{)}
\NormalTok{grafico }\OtherTok{=}\NormalTok{ grafico }\SpecialCharTok{+} \FunctionTok{labs}\NormalTok{(}\AttributeTok{color =} \StringTok{"Valores de df"}\NormalTok{)}

\FunctionTok{plot}\NormalTok{(grafico)}
\end{Highlighting}
\end{Shaded}

\includegraphics{03desafios_files/figure-latex/unnamed-chunk-5-1.pdf}

\begin{Shaded}
\begin{Highlighting}[]
\NormalTok{colors1 }\OtherTok{\textless{}{-}} \FunctionTok{c}\NormalTok{(}\StringTok{"df1 = 1"} \OtherTok{=} \StringTok{"blue4"}\NormalTok{, }\StringTok{"df2 = 2"} \OtherTok{=} \StringTok{"red4"}\NormalTok{, }\StringTok{"df3 = 5"} \OtherTok{=} \StringTok{"green4"}\NormalTok{, }\StringTok{"df4 = 1000000"} \OtherTok{=} \StringTok{"yellow3"}\NormalTok{)}

\NormalTok{x }\OtherTok{=} \FunctionTok{seq}\NormalTok{(}\SpecialCharTok{{-}}\DecValTok{4}\NormalTok{, }\DecValTok{4}\NormalTok{, }\AttributeTok{length =} \DecValTok{100}\NormalTok{)}
\NormalTok{distribucion }\OtherTok{=} \FunctionTok{pt}\NormalTok{(x, }\AttributeTok{df =} \DecValTok{1}\NormalTok{, }\AttributeTok{log =} \ConstantTok{FALSE}\NormalTok{)}
\NormalTok{distribucion2 }\OtherTok{=} \FunctionTok{pt}\NormalTok{(x, }\AttributeTok{df =} \DecValTok{2}\NormalTok{, }\AttributeTok{log =} \ConstantTok{FALSE}\NormalTok{)}
\NormalTok{distribucion5 }\OtherTok{=} \FunctionTok{pt}\NormalTok{(x, }\AttributeTok{df =} \DecValTok{5}\NormalTok{, }\AttributeTok{log =} \ConstantTok{FALSE}\NormalTok{)}
\NormalTok{distribucion10 }\OtherTok{=} \FunctionTok{pt}\NormalTok{(x, }\AttributeTok{df =} \DecValTok{1000000}\NormalTok{, }\AttributeTok{log =} \ConstantTok{FALSE}\NormalTok{)}

\NormalTok{datos}\OtherTok{=}\FunctionTok{data.frame}\NormalTok{(x,distribucion)}

\NormalTok{grafico }\OtherTok{=} \FunctionTok{ggplot}\NormalTok{(}\AttributeTok{data=}\NormalTok{datos,}\FunctionTok{aes}\NormalTok{(}\AttributeTok{x=}\NormalTok{x,}\AttributeTok{y=}\NormalTok{distribucion))}
\NormalTok{grafico }\OtherTok{=}\NormalTok{ grafico }\SpecialCharTok{+} \FunctionTok{geom\_line}\NormalTok{(}\FunctionTok{aes}\NormalTok{(}\AttributeTok{y =}\NormalTok{ distribucion, }\AttributeTok{color =} \StringTok{"df1 = 1"}\NormalTok{), }\AttributeTok{stat=}\StringTok{"identity"}\NormalTok{, }\AttributeTok{linewidth =} \DecValTok{1}\NormalTok{)}
\NormalTok{grafico }\OtherTok{=}\NormalTok{ grafico }\SpecialCharTok{+} \FunctionTok{geom\_line}\NormalTok{(}\FunctionTok{aes}\NormalTok{(}\AttributeTok{y=}\NormalTok{distribucion2, }\AttributeTok{color =} \StringTok{"df2 = 2"}\NormalTok{), }\AttributeTok{stat=}\StringTok{"identity"}\NormalTok{, }\AttributeTok{linewidth =} \DecValTok{1}\NormalTok{)}
\NormalTok{grafico }\OtherTok{=}\NormalTok{ grafico }\SpecialCharTok{+} \FunctionTok{geom\_line}\NormalTok{(}\FunctionTok{aes}\NormalTok{(}\AttributeTok{y=}\NormalTok{distribucion5, }\AttributeTok{color =} \StringTok{"df3 = 5"}\NormalTok{), }\AttributeTok{stat=}\StringTok{"identity"}\NormalTok{, }\AttributeTok{linewidth =} \DecValTok{1}\NormalTok{)}
\NormalTok{grafico }\OtherTok{=}\NormalTok{ grafico }\SpecialCharTok{+} \FunctionTok{geom\_line}\NormalTok{(}\FunctionTok{aes}\NormalTok{(}\AttributeTok{y=}\NormalTok{distribucion10, }\AttributeTok{color =} \StringTok{"df4 = 1000000"}\NormalTok{), }\AttributeTok{stat=}\StringTok{"identity"}\NormalTok{, }\AttributeTok{linewidth =} \DecValTok{1}\NormalTok{)}

\NormalTok{grafico }\OtherTok{=}\NormalTok{ grafico }\SpecialCharTok{+} \FunctionTok{theme\_bw}\NormalTok{() }\SpecialCharTok{+} \FunctionTok{ggtitle}\NormalTok{(}\StringTok{"Probabilidad acumulada {-} Student"}\NormalTok{)}
\NormalTok{grafico }\OtherTok{=}\NormalTok{ grafico }\SpecialCharTok{+} \FunctionTok{xlab}\NormalTok{(}\StringTok{"X"}\NormalTok{) }\SpecialCharTok{+} \FunctionTok{ylab}\NormalTok{(}\StringTok{"Probabilidad"}\NormalTok{)}

\NormalTok{grafico }\OtherTok{=}\NormalTok{ grafico }\SpecialCharTok{+} \FunctionTok{labs}\NormalTok{(}\AttributeTok{color =} \StringTok{"Valores de df"}\NormalTok{, )}

\FunctionTok{plot}\NormalTok{(grafico)}
\end{Highlighting}
\end{Shaded}

\includegraphics{03desafios_files/figure-latex/unnamed-chunk-6-1.pdf}

\hypertarget{establecer-una-semilla-para-el-generador-de-nuxfameros-aleatorios-con-la-funciuxf3n-set.seed.-genere-10-100-y-1000-datos-aleatorios-utilizando-la-funciuxf3n-con-prefijo-r-grafique-un-histograma-para-cada-caso.-quuxe9-se-puede-observar}{%
\subsection{3. Establecer una semilla para el generador de números
aleatorios con la función set.seed(). Genere 10, 100 y 1000 datos
aleatorios utilizando la función con prefijo ``r'', grafique un
histograma para cada caso. ¿Qué se puede
observar?}\label{establecer-una-semilla-para-el-generador-de-nuxfameros-aleatorios-con-la-funciuxf3n-set.seed.-genere-10-100-y-1000-datos-aleatorios-utilizando-la-funciuxf3n-con-prefijo-r-grafique-un-histograma-para-cada-caso.-quuxe9-se-puede-observar}}

\hypertarget{chi-squared-2}{%
\subsubsection{Chi-squared}\label{chi-squared-2}}

\begin{Shaded}
\begin{Highlighting}[]
\FunctionTok{set.seed}\NormalTok{(}\DecValTok{10}\NormalTok{)}
\NormalTok{datos10 }\OtherTok{=} \FunctionTok{rchisq}\NormalTok{(}\DecValTok{10}\NormalTok{, }\AttributeTok{df =} \DecValTok{3}\NormalTok{)}

\FunctionTok{hist}\NormalTok{(datos10, }\AttributeTok{main =} \StringTok{"Histograma de 10 datos Chisquare"}\NormalTok{, }\AttributeTok{xlab =} \StringTok{"X"}\NormalTok{, }\AttributeTok{ylab =} \StringTok{"Frecuencia"}\NormalTok{)}
\end{Highlighting}
\end{Shaded}

\includegraphics{03desafios_files/figure-latex/unnamed-chunk-7-1.pdf}

\begin{Shaded}
\begin{Highlighting}[]
\FunctionTok{set.seed}\NormalTok{(}\DecValTok{100}\NormalTok{)}
\NormalTok{datos100 }\OtherTok{=} \FunctionTok{rchisq}\NormalTok{(}\DecValTok{100}\NormalTok{, }\AttributeTok{df =} \DecValTok{3}\NormalTok{)}

\FunctionTok{hist}\NormalTok{(datos100, }\AttributeTok{main =} \StringTok{"Histograma de 100 datos Chisquare"}\NormalTok{, }\AttributeTok{xlab =} \StringTok{"X"}\NormalTok{, }\AttributeTok{ylab =} \StringTok{"Frecuencia"}\NormalTok{)}
\end{Highlighting}
\end{Shaded}

\includegraphics{03desafios_files/figure-latex/unnamed-chunk-8-1.pdf}

\begin{Shaded}
\begin{Highlighting}[]
\FunctionTok{set.seed}\NormalTok{(}\DecValTok{1000}\NormalTok{)}
\NormalTok{datos1000 }\OtherTok{=} \FunctionTok{rchisq}\NormalTok{(}\DecValTok{1000}\NormalTok{, }\AttributeTok{df =} \DecValTok{3}\NormalTok{)}

\FunctionTok{hist}\NormalTok{(datos1000, }\AttributeTok{main =} \StringTok{"Histograma de 1000 datos Chisquare"}\NormalTok{, }\AttributeTok{xlab =} \StringTok{"X"}\NormalTok{, }\AttributeTok{ylab =} \StringTok{"Frecuencia"}\NormalTok{)}
\end{Highlighting}
\end{Shaded}

\includegraphics{03desafios_files/figure-latex/unnamed-chunk-9-1.pdf}

\hypertarget{inverse-gamma-2}{%
\subsubsection{Inverse Gamma}\label{inverse-gamma-2}}

\begin{Shaded}
\begin{Highlighting}[]
\FunctionTok{library}\NormalTok{(extraDistr)}
\FunctionTok{set.seed}\NormalTok{(}\DecValTok{10}\NormalTok{)}
\NormalTok{datos10 }\OtherTok{=} \FunctionTok{rinvgamma}\NormalTok{(}\DecValTok{10}\NormalTok{, }\AttributeTok{alpha =} \DecValTok{3}\NormalTok{, }\AttributeTok{beta =} \FloatTok{0.5}\NormalTok{)}

\FunctionTok{hist}\NormalTok{(datos10, }\AttributeTok{main =} \StringTok{"Histograma de 10 datos Gamma Inversa"}\NormalTok{, }\AttributeTok{xlab =} \StringTok{"X"}\NormalTok{, }\AttributeTok{ylab =} \StringTok{"Frecuencia"}\NormalTok{)}
\end{Highlighting}
\end{Shaded}

\includegraphics{03desafios_files/figure-latex/unnamed-chunk-10-1.pdf}

\begin{Shaded}
\begin{Highlighting}[]
\FunctionTok{set.seed}\NormalTok{(}\DecValTok{100}\NormalTok{)}
\NormalTok{datos100 }\OtherTok{=} \FunctionTok{rinvgamma}\NormalTok{(}\DecValTok{100}\NormalTok{, }\AttributeTok{alpha =} \DecValTok{3}\NormalTok{, }\AttributeTok{beta =} \FloatTok{0.5}\NormalTok{)}

\FunctionTok{hist}\NormalTok{(datos100, }\AttributeTok{main =} \StringTok{"Histograma de 100 datos Gamma Inversa"}\NormalTok{, }\AttributeTok{xlab =} \StringTok{"X"}\NormalTok{, }\AttributeTok{ylab =} \StringTok{"Frecuencia"}\NormalTok{)}
\end{Highlighting}
\end{Shaded}

\includegraphics{03desafios_files/figure-latex/unnamed-chunk-11-1.pdf}

\begin{Shaded}
\begin{Highlighting}[]
\FunctionTok{set.seed}\NormalTok{(}\DecValTok{1000}\NormalTok{)}
\NormalTok{datos1000 }\OtherTok{=} \FunctionTok{rinvgamma}\NormalTok{(}\DecValTok{1000}\NormalTok{, }\AttributeTok{alpha =} \DecValTok{3}\NormalTok{, }\AttributeTok{beta =} \FloatTok{0.5}\NormalTok{)}

\FunctionTok{hist}\NormalTok{(datos1000, }\AttributeTok{main =} \StringTok{"Histograma de 1000 datos Gamma Inversa"}\NormalTok{, }\AttributeTok{xlab =} \StringTok{"X"}\NormalTok{, }\AttributeTok{ylab =} \StringTok{"Frecuencia"}\NormalTok{)}
\end{Highlighting}
\end{Shaded}

\includegraphics{03desafios_files/figure-latex/unnamed-chunk-12-1.pdf}

\hypertarget{student-2}{%
\subsubsection{Student}\label{student-2}}

\begin{Shaded}
\begin{Highlighting}[]
\FunctionTok{set.seed}\NormalTok{(}\DecValTok{10}\NormalTok{)}
\NormalTok{datos10 }\OtherTok{=} \FunctionTok{rt}\NormalTok{(}\DecValTok{10}\NormalTok{, }\AttributeTok{df =} \DecValTok{10000000}\NormalTok{)}

\FunctionTok{hist}\NormalTok{(datos10, }\AttributeTok{main =} \StringTok{"Histograma de 10 datos Student"}\NormalTok{, }\AttributeTok{xlab =} \StringTok{"X"}\NormalTok{, }\AttributeTok{ylab =} \StringTok{"Frecuencia"}\NormalTok{)}
\end{Highlighting}
\end{Shaded}

\includegraphics{03desafios_files/figure-latex/unnamed-chunk-13-1.pdf}

\begin{Shaded}
\begin{Highlighting}[]
\FunctionTok{set.seed}\NormalTok{(}\DecValTok{100}\NormalTok{)}
\NormalTok{datos100 }\OtherTok{=} \FunctionTok{rt}\NormalTok{(}\DecValTok{100}\NormalTok{, }\AttributeTok{df =} \DecValTok{10000000}\NormalTok{)}

\FunctionTok{hist}\NormalTok{(datos100, }\AttributeTok{main =} \StringTok{"Histograma de 100 datos Student"}\NormalTok{, }\AttributeTok{xlab =} \StringTok{"X"}\NormalTok{, }\AttributeTok{ylab =} \StringTok{"Frecuencia"}\NormalTok{)}
\end{Highlighting}
\end{Shaded}

\includegraphics{03desafios_files/figure-latex/unnamed-chunk-14-1.pdf}

\begin{Shaded}
\begin{Highlighting}[]
\FunctionTok{set.seed}\NormalTok{(}\DecValTok{1000}\NormalTok{)}
\NormalTok{datos1000 }\OtherTok{=} \FunctionTok{rt}\NormalTok{(}\DecValTok{1000}\NormalTok{, }\AttributeTok{df =} \DecValTok{10000000}\NormalTok{)}

\FunctionTok{hist}\NormalTok{(datos1000, }\AttributeTok{main =} \StringTok{"Histograma de 1000 datos Student"}\NormalTok{, }\AttributeTok{xlab =} \StringTok{"X"}\NormalTok{, }\AttributeTok{ylab =} \StringTok{"Frecuencia"}\NormalTok{)}
\end{Highlighting}
\end{Shaded}

\includegraphics{03desafios_files/figure-latex/unnamed-chunk-15-1.pdf}

\hypertarget{conclusiuxf3n-general}{%
\subsection{Conclusión general}\label{conclusiuxf3n-general}}

Al analizar los histogramas de cada distribución con datos aleatorios,
se puede visualizar que mientras más datos se generenn, mayor será la
semejanza entre la forma del histograma y el gráfico de las
probabilidades de cada distribución.

Es decir, mientras más grande la muestra, se representará de mejor forma
la población.

\hypertarget{refs}{}
\begin{CSLReferences}{1}{0}
\leavevmode\vadjust pre{\hypertarget{ref-cerda2007interpretacion}{}}%
Cerda, J., \& VILLARROEL DEL, L. (2007). Interpretaci{ó}n del test de
chi-cuadrado (x\(^2\)) en investigaci{ó}n pedi{á}trica. \emph{Revista
Chilena de Pediatr{ı́}a}, \emph{78}(4), 414--417.

\leavevmode\vadjust pre{\hypertarget{ref-GAO2016216}{}}%
Gao, C.-T., \& Zhou, X.-H. (2016). Forecasting VaR and ES using dynamic
conditional score models and skew student distribution. \emph{Economic
Modelling}, \emph{53}, 216--223.
https://doi.org/\url{https://doi.org/10.1016/j.econmod.2015.12.004}

\leavevmode\vadjust pre{\hypertarget{ref-doi:10.1080ux2f03610926.2013.768667}{}}%
Mead, M. E. (2015). Generalized inverse gamma distribution and its
application in reliability. \emph{Communications in Statistics - Theory
and Methods}, \emph{44}(7), 1426--1435.
\url{https://doi.org/10.1080/03610926.2013.768667}

\end{CSLReferences}

\end{document}
